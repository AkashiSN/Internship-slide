\documentclass[unicode,18px,aspectratio=169]{beamer} %アスペクト比16:9

%% 和文用 %%
\usepackage{bxdpx-beamer}
\usepackage{zxjatype}
\setCJKmainfont[Scale=0.95]{Meiryo} % フォント埋め込み

%% スライドのテーマ %%
%% https://qiita.com/htlsne/items/70cbb488e7a87cd9e228
\usetheme[progressbar=frametitle]{metropolis}

%% 各パッケージ %%
\usepackage{amsmath} % 数式用
\usepackage{bm} % ボールドイタリック(ベクトル)
\usepackage{color} % 文字に色をつける
\usepackage{slashbox} % 表に斜め線をつける
\usepackage{hyperref} % ハイパーリンク
\hypersetup{urlcolor=cyan} % URLの色をシアンにする
\usepackage{url} % URL

%% タイトル %%
\title[インターンシップ報告会]{インターンシップ報告会2018}%[略タイトル]{タイトル}
\author[西]{西\ 総一朗}%[略名前]{名前}
\institute[4EJ]{明石高専4年電気情報工学科}%[略所属]{所属}
\newcommand{\todaye}{October 25, 2018}
\date{\todaye}%日付

\begin{document}
	\begin{frame}[plain]\frametitle{}
		\titlepage %表紙
	\end{frame}

	\section{インターンシップの概要} % (fold)
	\label{sec:インターンシップの概要}
		\begin{frame}\frametitle{インターンシップの概要}
			\begin{table}[]
				\begin{tabular}{ll}
				\bf{受け入れ先} & 合同会社ヘマタイト \\
				\bf{受け入れ期間} & 1ヶ月 \\
				\bf{実施場所} & 本社(東京) \\
				\bf{実施内容} & 製品の問題調査・問題解決・開発業務 \\
						& サーバー保守\ \ カスタマーサポート \\
						& セキュリティコンテストの運営業務
				\end{tabular}
			\end{table}
		\end{frame}
	% section インターンシップの概要 (end)

	\section{会社概要} % (fold)
	\label{sec:会社概要}
		\begin{frame}\frametitle{合同会社ヘマタイト}
			\begin{block}{}
				\begin{columns}[c]
					\begin{column}{0.4\textwidth}
					\includegraphics[width=\columnwidth]{image/logo_with_catchphrase.png}
					\end{column}
				\end{columns}
			\end{block}
			\begin{alertblock}{}
				\begin{columns}[t]
					\begin{column}{0.55\textwidth}
						\begin{block}{ビジョン}
							\begin{itemize}
								\item 「人とコンピュータの良い関係を作る。」
							\end{itemize}
						\end{block}
						\begin{block}{事業領域}
							\begin{itemize}
								\item ソフトウェア開発
								\item 情報セキュリティ
								\item 情報教育
							\end{itemize}
						\end{block}
					\end{column}
					\begin{column}{0.45\textwidth}
						\begin{block}{ソフトウェア開発}
							\begin{itemize}
								\item CTF\,Kit
								\includegraphics[width=0.7\textwidth]{image/ctfkit-logo-small.png}
								\item その他受託
							\end{itemize}
						\end{block}
					\end{column}
				\end{columns}
			\end{alertblock}
		\end{frame}

		\begin{frame}\frametitle{CTF\,Kitとは}
			\begin{block}{}
				\begin{columns}[c]
					\begin{column}{0.4\textwidth}
					\includegraphics[width=\columnwidth]{image/ctfkit-logo-small.png}
					\end{column}
				\end{columns}
			\end{block}
			\begin{alertblock}{CTFの準備、これひとつ。}
				\begin{itemize}
					\item CTF(Capture The Flag)を通じて、情報セキュリティ教育を行うための教材
				\end{itemize}
			\end{alertblock}
			\begin{block}{導入実績}
				学校教育から企業研修の様々なシーンで活用されているほか、「KOSENセキュリティコンテスト」\footnote{\url{https://sckosen2018.sasebo.ac.jp/index.html}}でも採用実績があります。
			\end{block}
		\end{frame}
	% section 会社概要 (end)

	\begin{frame}{CTFとは(1/2)}
		\begin{alertblock}{CTFとは}
			\begin{itemize}
				\item CTFとはCapture The Flagの略で、情報セキュリティ技術の競技方式の一つ
				\item 問題に隠されたFlagを様々な技術を使って探し出すことで得点することができ、その得点を競う競技
				\item ゲーム形式であるため集中して取り組むことができ、参加者がセキュリティスキルを楽しみながら効率的に高めることができる
			\end{itemize}
		\end{alertblock}
	\end{frame}

	\begin{frame}{CTFとは(2/2)}
		\begin{block}{情報セキュリティ人材の不足}
			経済産業省が2016年に発表した「IT人材の最新動向と将来推計に関する調査」\footnote{\url{http://www.meti.go.jp/policy/it_policy/jinzai/27FY/ITjinzai_report_summary.pdf}}でも、
			情報セキュリティ人材が産業界で不足していることが指摘されています。
		\end{block}
		\begin{block}{国内でもCTFへの注目高まる}
			「SECCON」\footnote{\url{https://2018.seccon.jp/}}や「富士通サイバーセキュリティコンテスト」\footnote{\url{http://www.fujitsu.com/jp/group/fnets/solutions/cybersecurity/edutraservice/}}など、国際大会を含めた大型イベントが国内で続々と開催されています。
			CTFは人材獲得に困難を抱える企業の一つの解答となりつつあり、解決手段としての注目も高まっています。
		\end{block}
	\end{frame}

	\section{インターンシップでの活動内容} % (fold)
	\label{sec:インターンシップでの活動内容}
	\begin{frame}\frametitle{インターンシップでの活動内容}
		\begin{alertblock}{活動内容}
			\begin{itemize}
				\item CTF\,Kitの開発業務
				\item セキュリティコントストの運営業務
				\item コンテストに出題される問題作成
				\item コンテストのインフラ整備(サーバー・競技ネットワークの構築)
			\end{itemize}
		\end{alertblock}
	\end{frame}

	\begin{frame}\frametitle{CTF\,Kitの開発業務}
		\begin{alertblock}{APIサーバーの開発}
			\begin{itemize}
				\item Golangを用いたAPIサーバーの開発
					\begin{itemize}
						\item 新規機能追加
						\item 問題解決
					\end{itemize}
				\item Packer,\,Ansibleを用いたデプロイ環境の構築
			\end{itemize}
			\begin{columns}[c]
				\begin{column}{0.4\textwidth}
					\includegraphics[width=\columnwidth]{image/commits.png}
				\end{column}
			\end{columns}
		\end{alertblock}
	\end{frame}

	\begin{frame}\frametitle{セキュリティコンテストの運営業務(1/2)}
		\begin{alertblock}{KOSENセキュリティコンテスト2018}
			KSEC\footnote{独立行政法人国立高等専門学校機構\ 情報セキュリティ人材育成事業}の一環として行われているコンテスト
			\begin{columns}[c]
				\begin{column}{0.5\textwidth}
					\begin{itemize}
						\item 今年で3回目
						\item 佐世保高専が主催校
					\end{itemize}
					\vspace{2mm}
					インターンシップで行った内容
					\begin{itemize}
						\item 出題される問題作成
						\item 競技環境の構築
						\item 競技中のサーバー保守
						\item 競技中のカスタマーサポート
					\end{itemize}
				\end{column}
				\begin{column}{0.45\textwidth}
					\includegraphics[width=\columnwidth]{image/sckosen2018.jpg}
				\end{column}
			\end{columns}
		\end{alertblock}
	\end{frame}

	\begin{frame}\frametitle{セキュリティコンテストの運営業務(2/2)}
		\begin{alertblock}{某大手通信企業様でのCTF大会}
			\begin{columns}[c]
				\begin{column}{0.5\textwidth}
					\begin{itemize}
						\item 社員研修としてCTF大会
					\end{itemize}
					\vspace{5mm}
					インターンシップで行った内容
					\begin{itemize}
						\item 出題される問題作成
						\item 競技環境の構築
						\item 競技中のサーバー保守
						\item 競技中のカスタマーサポート
					\end{itemize}
				\end{column}
				\begin{column}{0.5\textwidth}
					\includegraphics[width=\columnwidth]{image/nwday.jpg}
				\end{column}
			\end{columns}
		\end{alertblock}
	\end{frame}

	\begin{frame}\frametitle{インターンシップを通じて学んだこと}
		\begin{alertblock}{学んだこと}
			\begin{itemize}
				\item チーム開発スキル
				\item セキュリティ業界の実情
				\item 大手企業におけるセキュリティに対する姿勢
				\item 高専機構としてのセキュリティ教育
			\end{itemize}
		私が将来就きたい業界であるセキュリティに関することを多く学ぶことができました。
		\end{alertblock}
	\end{frame}
	% section インターンシップでの活動内容 (end)

	\section{ありがとうございました} % (fold)
	\label{sec:ありがとうございました}
	% section ありがとうございました (end)
\end{document}